%************************************************
\chapter{MHM-Systemhaus GmbH}\label{ch:mhm}
%************************************************

\section{History}
MHM-Systemhaus GmbH (MHM) is a German IT Company based in Stuttgart that was founded on April \nth{1}, 2001, by Armin Mendle, Christian Hasenstab and Steffen Michel. Currently, only Mendle and Michel manage the firm, due to Hasenstab's departure in 2005 as the company switched from a GbR to a GmbH\marginpar{GbR \& GmbH are types of legal entities in Germany.}

MHM develops customized Talent Management Software\footnote{\url{http://en.wikipedia.org/wiki/Talent_management_system}}, eRecruiting Applications and ERA-Software\footnote{EntgeltRahmenAbkommen is a regulated compensation agreement in the Metal \& Electronic Industry}. These solutions are developed as Web Applications and are adaptable to existing HR-Processes\marginpar{HR: Human Resources}.

Two of the biggest advantages of the MHM Software Solutions are the priority it gives to data protection and the high adjustability to processes that the clients already have in place. Because of this, MHM was audited in 2010 and certified to comply with German Data Protection Law (\textsection{11} BDSG).\cite{michel:2012}

Until 2011, development of the MHM eRecruiting applications was done using an  Java Framework developed in-house that is still maintained for older clients, but no longer updated. Since then, the team has moved the entire eRecruiting System to Ruby on Rails\footnote{\url{http://rubyonrails.org}}, an object oriented, web development framework based on the \ac{MVC} architecture pattern. The new framework gives the development team a high amount of flexibility when customizing the application and provides a level of abstraction that assures a faster learning curve for new developers. The reason for this complete rewrite was to have an application that is future-proof, capable of being maintained and upgraded for years to come.

\section{MHM eRecruiting}

The Centre of Human Resources Information Systems of the University of Bamberg and Frankfurt am Main in conjunction with Monster Worldwide Deutschland GmbH surveyed 1000 small- and medium-sized Companies in Germany about their Recruiting procedures in 2012.\cite[p. 6]{weitzel:2012}

The study showed that 37.5\% of all new recruits came from job postings either listed on the company's website or from Online Job Markets. The percentage of paper-based Applications has diminished constantly for the past few years and 2012 was the first year that companies preferred applications via E-Mail. Some companies even predict that by the year 2016 less than a third of all applications will be paper based. Social Media plays a huge roll in this change. 48.3\% of the surveyed companies see the possibility of reaching more candidates by using Facebook, LinkedIn or XING.\cite[p. 8]{weitzel:2012}


\begin{quotation}
eRecruiting is the process of personnel recruitment using electronic resources, in particular the internet. Using database technologies, and online job advertising boards and search engines, employers can now fill posts in a fraction of the time previously possible. Using an online e-Recruitment system may potentially save the employer time as usually they can rate the eCandidate and several persons in HR independently review eCandidates.
\cite{wikipedia:erec}

\end{quotation}


The MHM eRecruiting Application can be used for paper-based applications as well as online applications. It offers the possibility to publish any vacant position to a number of different channels. It can be automatically posted on Facebook or XING, or on external Job Portals like Stepstone and also on the customer's web site.

This level of automation and practicality is what has made the eRecruiting solution so successful. We want to continue offering more options to customers, which is why the main goal of this thesis is to develop a Mobile Application that acts as another Job Portal Channel and aggregates vacant positions, from customers that wish to use the feature, into an easy to use, always available, always online, mobile platform. %thus giving more exposure to each vacancy and 

Before continuing, let's explain some of the technical terms regarding an eRecruiting System that will be used throughout this work.

\begin{description}
\item[Company] This term refers to one of the clients of \textit{MHM} that uses the eRecruiting System.
\item[Vacancy] An empty position inside a company.
\item[Publication] A vacancy that has been made available to the public and has been published via the eRecruiting System.
\item[Job Portal] An application or website that collects open Publications and organizes everything in a single, easy to use, searchable place.
\end{description}


















 
   
