%************************************************
\chapter{Introduction}\label{ch:introduction}
%************************************************

\paragraph{Important Note:} Some things of this style might look
unusual at first glance, many people feel so in the beginning.
However, all things are intentionally designed to be as they are,
especially these:




\begin{lstlisting}[frame=lt]
% **************************************************
% 2. Personal data and user ad-hoc commands
% **************************************************
\newcommand{\myTitle}{A Classic Thesis Style\xspace} 
\newcommand{\mySubtitle}{An Homage to...\xspace} 
\end{lstlisting}

Further customization can be made in \texttt{classicthesis-config.tex}
by choosing the options to \texttt{classicthesis.sty} 
(see~\autoref{sec:options}) in a line that looks like this:

\begin{lstlisting}[frame=lt]
\PassOptionsToPackage{eulerchapternumbers,drafting,listings,subfig,eulermath,parts}{classicthesis}
\end{lstlisting}

If you want to use backreferences from your citations to the pages
they were cited on, change the following line from:
\begin{lstlisting}[breaklines=false,frame=lt]
\setboolean{enable-backrefs}{false} % true false
\end{lstlisting}
to
\begin{lstlisting}[breaklines=false,frame=lt]
\setboolean{enable-backrefs}{true} % true false
\end{lstlisting}

This can be solved by passing \texttt{es-tabla} parameter to \texttt{babel}:
\begin{verbatim}
	\PassOptionsToPackage{es-tabla,spanish,es-lcroman,english}{babel}
	\usepackage{babel}
\end{verbatim}

