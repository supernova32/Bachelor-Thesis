%************************************************
\chapter{Introduction}\label{ch:introduction}
%************************************************

\paragraph{Important Note:} Some things of this style might look
unusual at first glance, many people feel so in the beginning.
However, all things are intentionally designed to be as they are,
especially these:
\begin{itemize}
    \item No bold fonts are used. Italics or spaced small caps do the
    job quite well.
    \item The size of the text body is intentionally shaped like it
    is. It supports both legibility and allows a reasonable amount of
    information to be on a page. And, no: the lines are not too short.
    \item The tables intentionally do not use vertical or double
    rules. See the documentation for the \texttt{booktabs} package for
    a nice discussion of this topic.\footnote{To be found online at \\
    \url{http://www.ctan.org/tex-archive/macros/latex/contrib/booktabs/}.}
    \item And last but not least, to provide the reader with a way
    easier access to page numbers in the table of contents, the page
    numbers are right behind the titles. Yes, they are \emph{not}
    neatly aligned at the right side and they are \emph{not} connected
    with dots that help the eye to bridge a distance that is not
    necessary. If you are still not convinced: is your reader
    interested in the page number or does she want to sum the numbers
    up?
\end{itemize}
Therefore, please do not break the beauty of the style by changing
these things unless you really know what you are doing! Please.


\section{Organization}
A very important factor for successful thesis writing is the
organization of the material. This template suggests a structure as
the following:
\begin{itemize}
    \marginpar{You can use these margins for summaries of the text
    body\dots}
    \item\texttt{Chapters/} is where all the ``real'' content goes in
    separate files such as \texttt{Chapter01.tex} etc.
 %  \item\texttt{Examples/} is where you store all listings and other
 %  examples you want to use for your text.
    \item\texttt{FrontBackMatter/} is where all the stuff goes that
    surrounds the ``real'' content, such as the acknowledgments,
    dedication, etc.
    \item\texttt{gfx/} is where you put all the graphics you use in
    the thesis. Maybe they should be organized into subfolders
    depending on the chapter they are used in, if you have a lot of
    graphics.
    \item\texttt{Bibliography.bib}: the Bib\TeX\ database to organize
    all the references you might want to cite.
    \item\texttt{classicthesis.sty}: the style definition to get this
    awesome look and feel. Does not only work with this thesis template
    but also on its own (see folder \texttt{Examples}). Bonus: works
    with both \LaTeX\ and \textsc{pdf}\LaTeX\dots and \mLyX.
    \item\texttt{ClassicThesis.tcp} a \TeX nicCenter project file.
    Great tool and it's free!
    \item\texttt{ClassicThesis.tex}: the main file of your thesis
    where all gets bundled together.
    \item\texttt{classicthesis-config.tex}: a central place to load all 
    nifty packages that are used. In there, you can also activate 
    backrefs in order to have information in the bibliography about 
    where a source was cited in the text (\ie, the page number).
    
    \emph{Make your changes and adjustments here.} This means that you  
    specify here the options you want to load \texttt{classicthesis.sty} 
    with. You also adjust the title of your thesis, your name, and all 
    similar information here. Refer to \autoref{sec:custom} for more 
    information.
    
		This had to change as of version 3.0 in order to enable an easy 
		transition from the ``basic'' style to \mLyX.
    
\end{itemize}
In total, this should get you started in no time.


\section{Style Options}\label{sec:options}
There are a couple of options for \texttt{classicthesis.sty} that
allow for a bit of freedom concerning the layout:
\marginpar{\dots or your supervisor might use the margins for some
    comments of her own while reading.}
\begin{itemize}
	\item General:
		\begin{itemize}
			\item\texttt{drafting}: prints the date and time at the bottom of
    each page, so you always know which version you are dealing with.
    Might come in handy not to give your Prof. that old draft.
		\end{itemize}
	
	\item Parts and Chapters:
		\begin{itemize}
			\item\texttt{parts}: if you use Part divisions for your document,
    you should choose this option. (Cannot be used together with 
    \texttt{nochapters}.)
    
			\item\texttt{nochapters}: allows to use the look-and-feel with 
    classes that do not use chapters, \eg, for articles. Automatically
    turns off a couple of other options: \texttt{eulerchapternumbers}, 
    \texttt{linedheaders}, \texttt{listsseparated}, and \texttt{parts}. 
    
	    \item\texttt{linedheaders}: changes the look of the chapter
	    headings a bit by adding a horizontal line above the chapter
	    title. The chapter number will also be moved to the top of the
	    page, above the chapter title.
    
		\end{itemize}

  \item Typography:
		\begin{itemize}
				\item\texttt{eulerchapternumbers}: use figures from Hermann Zapf's
    Euler math font for the chapter numbers. By default, old style
    figures from the Palatino font are used.
    
        \item\texttt{beramono}: loads Bera Mono as typewriter font. 
    (Default setting is using the standard CM typewriter font.)
    \item\texttt{eulermath}: loads the awesome Euler fonts for math. 
    (Palatino is used as default font.)
    
		    \item\texttt{pdfspacing}: makes use of pdftex' letter spacing
		    capabilities via the \texttt{microtype} package.\footnote{Use 
		    \texttt{microtype}'s \texttt{DVIoutput} option to generate
		    DVI with pdftex.} This fixes some serious issues regarding 
		    math formul\ae\ etc. (\eg, ``\ss'') in headers. 
		    
		    \item\texttt{minionprospacing}: uses the internal \texttt{textssc}
		    command of the \texttt{MinionPro} package for letter spacing. This 
		    automatically enables the \texttt{minionpro} option and overrides
		    the \texttt{pdfspacing} option.
    
		\end{itemize}  

	\item Table of Contents:
		\begin{itemize}
			 \item\texttt{tocaligned}: aligns the whole table of contents on
		    the left side. Some people like that, some don't.
		    
		    \item\texttt{dottedtoc}: sets pagenumbers flushed right in the 
		    table of contents.

			\item\texttt{manychapters}: if you need more than nine chapters for 
	    your document, you might not be happy with the spacing between the 
	    chapter number and the chapter title in the Table of Contents. 
	    This option allows for additional space in this context. 
	    However, it does not look as ``perfect'' if you use
	    \verb|\parts| for structuring your document.
		    
		\end{itemize}
    
	\item Floats:
		\begin{itemize}
    \item\texttt{listings}: loads the \texttt{listings} package (if not 
    already done) and configures the List of Listings accordingly.
    
    \item\texttt{floatperchapter}: activates numbering per chapter for
    all floats such as figures, tables, and listings (if used).	
    
	    \item\texttt{subfig}(\texttt{ure}): is passed to the \texttt{tocloft} 
	    package to enable compatibility with the \texttt{subfig}(\texttt{ure}) 
	    package. Use this option if you want use classicthesis with the
	    \texttt{subfig} package.
    	
%    \item\texttt{listsseparated}: will add extra space between table
%    and figure entries of different chapters in the list of tables or
%    figures, respectively. % Deprecated as of version 2.9.
		\end{itemize}    
 
% 	\item\texttt{a5paper}: adjusts the page layout according to the
%    global \texttt{a5paper} option (\emph{experimental} feature).
%    \item\texttt{minionpro}: sets Robert Slimbach's Minion as the 
%    main font of the document. The textblock size is adjusted 
%    accordingly.    

   \end{itemize}
The best way to figure these options out is to try the different
possibilities and see, what you and your supervisor like best.

In order to make things easier in general, 
\texttt{classicthesis-config.tex} 
contains some useful commands that might help you.


\section{Customization}\label{sec:custom}
%(As of v3.0, the Classic Thesis Style for \LaTeX{} and \mLyX{} share
%the same two \texttt{.sty} files.)
This section will give you some hints about how to adapt 
\texttt{classicthesis} to your needs.

The file \texttt{classicthesis.sty}
contains the core functionality of the style and in most cases will
be left intact, whereas the file \texttt{classic\-thesis-config.tex}
is used for some common user customizations. 

The first customization you are about to make is to alter the document
title, author name, and other thesis details. In order to do this, replace
the data in the following lines of \texttt{classicthesis-config.tex:}%
\marginpar{Modifications in \texttt{classic\-thesis-config.tex}%
}

\begin{lstlisting}[frame=lt]
% **************************************************
% 2. Personal data and user ad-hoc commands
% **************************************************
\newcommand{\myTitle}{A Classic Thesis Style\xspace} 
\newcommand{\mySubtitle}{An Homage to...\xspace} 
\end{lstlisting}

Further customization can be made in \texttt{classicthesis-config.tex}
by choosing the options to \texttt{classicthesis.sty} 
(see~\autoref{sec:options}) in a line that looks like this:

\begin{lstlisting}[frame=lt]
\PassOptionsToPackage{eulerchapternumbers,drafting,listings,subfig,eulermath,parts}{classicthesis}
\end{lstlisting}

If you want to use backreferences from your citations to the pages
they were cited on, change the following line from:
\begin{lstlisting}[breaklines=false,frame=lt]
\setboolean{enable-backrefs}{false} % true false
\end{lstlisting}
to
\begin{lstlisting}[breaklines=false,frame=lt]
\setboolean{enable-backrefs}{true} % true false
\end{lstlisting}

Many other customizations in \texttt{classicthesis-config.tex} are
possible, but you should be careful making changes there, since some
changes could cause errors.

Finally, changes can be made in the file \texttt{classicthesis.sty},%
\marginpar{Modifications in \texttt{classicthesis.sty}%
} although this is mostly not designed for user customization. The
main change that might be made here is the text-block size, for example,
to get longer lines of text.


\section{Issues}\label{sec:issues}
This section will list some information about problems using
\texttt{classic\-thesis} in general or using it with other packages.

Beta versions of \texttt{classicthesis} can be found at the following 
Google code repository:
\begin{center}
	\url{http://code.google.com/p/classicthesis/}
\end{center}
There, you can also post serious bugs and problems you encounter.

\subsection*{Compatibility with the \texttt{glossaries} Package}
If you want to use the \texttt{glossaries} package, take care of loading it 
with the following options:
\begin{verbatim}
	\usepackage[style=long,nolist]{glossaries}
\end{verbatim}
Thanks to Sven Staehs for this information. 


\subsection*{Compatibility with the (Spanish) \texttt{babel} Package}
Spanish languages need an extra option in order to work with this template:
\begin{verbatim}
	\usepackage[spanish,es-lcroman]{babel}
\end{verbatim}
Thanks to an unknown person for this information (via Google Code issue reporting). 


\paragraph{Further information for using \texttt{classicthesis} with Spanish (in addition to the above)}
In the file \texttt{ClassicThesis.tex} activate the language: 
\begin{verbatim}
	\selectlanguage{spanish}
\end{verbatim}
	
In order to get the bibliography style right, you can use the following:
\begin{verbatim}
	\bibliographystyle{babplain}
\end{verbatim}

For this, it is necessary to load the package:
\begin{verbatim}
	\usepackage[spanish,fixlanguage]{babelbib}
	\selectbiblanguage{spanish}
\end{verbatim}

If there are issues changing \verb|\tablename|, \eg, using this:
\begin{verbatim}
	\renewcommand{\bibname}{Referencias}
	\renewcommand{\tablename}{Tabla}
\end{verbatim}

This can be solved by passing \texttt{es-tabla} parameter to \texttt{babel}:
\begin{verbatim}
	\PassOptionsToPackage{es-tabla,spanish,es-lcroman,english}{babel}
	\usepackage{babel}
\end{verbatim}

But it is also necessary set \texttt{spanish} in the \verb|\documentclass|.

Thanks to Alvaro Jaramillo Duque for this information. 



